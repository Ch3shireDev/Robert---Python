\documentclass[11pt]{article}
\usepackage[utf8]{inputenc}
\usepackage{polski} 
\usepackage{geometry}
\geometry{a4paper}
\usepackage{graphicx}
\usepackage{booktabs}
\usepackage{array}
\usepackage{paralist}
\usepackage{verbatim}
\usepackage{subfig}
\usepackage{amsmath}
\usepackage{float}
\usepackage{amsthm}
\usepackage{amssymb}
\usepackage{amsfonts}
\usepackage{float}
\usepackage{thmtools}
\theoremstyle{definition}
\newtheorem{zad}{Zadanie}
\numberwithin{zad}{section}
\newtheorem{theorem}{Twierdzenie}
\newtheorem{definition}{Definicja}
\newtheorem{lemma}{Lemat}
\renewcommand*{\proofname}{Rozwiązanie}

\usepackage{fancyhdr}
\pagestyle{fancy}
\renewcommand{\headrulewidth}{0pt}
\usepackage{sectsty}
\allsectionsfont{\sffamily\mdseries\upshape}
\usepackage[nottoc,notlof,notlot]{tocbibind}
\usepackage[titles,subfigure]{tocloft}
\renewcommand{\cftsecfont}{\rmfamily\mdseries\upshape}
\renewcommand{\cftsecpagefont}{\rmfamily\mdseries\upshape}
\newcommand{\blank}{\framebox{\phantom{xxx}} }
\title{Fizyka}

\begin{document}
\maketitle
\tableofcontents

\section{Dynamika}
\subsection{Zasady dynamiki Newtona}

Zacznijmy od wymienienia wszystkich trzech zasad:

\begin{enumerate}
\item Ciało na którą działa zerowa siła wypadkowa porusza się bez przyspieszeń.
\item Przyspieszenie ciała jest wprost proporcjonalne do wypadkowej siły nań działającej.
\item Oddziaływania ciał są zawsze wzajemne. Suma oddziaływań w układach inercjalnych zawsze się zeruje.
\end{enumerate}

\subsection{Rodzaje oddziaływań}

Nowoczesna fizyka rozróżnia cztery podstawowe oddziaływania:

\begin{enumerate}
\item Grawitacja,
\item Elektromagnetyzm,
\item Oddziaływania silne,
\item Oddziaływania słabe.
\end{enumerate}

Ostatnie dwa odpowiadają za oddziaływania wewnątrz atomowe - dzięki oddziaływaniom silnym kwarki trzymają się razem i tworzą neutrony i protony, które następnie tworzą jądra atomów. Oddziaływania słabe są znacznie bardziej subtelne i występują w przemianach kwarków i leptonów. Dwa pierwsze oddziaływania znamy znacznie lepiej z życia - grawitacja jest oddziaływaniem mas na siebie, elektromagnetyzm jest oddziaływaniem ładunków elektrycznych na siebie.

\subsection{...inne oddziaływania?}

Na lekcjach fizyki poznamy inne siły - siłę tarcia, siłę oporu, siłę wyporu, siłę sprężystości, etc etc. Nie są to manifestacje nowych oddziaływań, tylko doświadczalne przejawy znanych efektów.


\end{document}
