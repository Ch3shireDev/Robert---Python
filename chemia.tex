\documentclass[11pt]{article}
\usepackage[utf8]{inputenc}
\usepackage{polski} 
\usepackage{geometry}
\geometry{a4paper}
\usepackage{graphicx}
\usepackage{booktabs}
\usepackage{array}
\usepackage{paralist}
\usepackage{verbatim}
\usepackage{subfig}
\usepackage{amsmath}
\usepackage{float}
\usepackage{amsthm}
\usepackage{amssymb}
\usepackage{amsfonts}
\usepackage{float}
\usepackage{thmtools}
\theoremstyle{definition}
\newtheorem{zad}{Zadanie}
\numberwithin{zad}{section}
\newtheorem{theorem}{Twierdzenie}
\newtheorem{definition}{Definicja}
\newtheorem{lemma}{Lemat}
\renewcommand*{\proofname}{Rozwiązanie}

\usepackage{fancyhdr}
\pagestyle{fancy}
\renewcommand{\headrulewidth}{0pt}
\usepackage{sectsty}
\allsectionsfont{\sffamily\mdseries\upshape}
\usepackage[nottoc,notlof,notlot]{tocbibind}
\usepackage[titles,subfigure]{tocloft}
\renewcommand{\cftsecfont}{\rmfamily\mdseries\upshape}
\renewcommand{\cftsecpagefont}{\rmfamily\mdseries\upshape}
\newcommand{\blank}{\framebox{\phantom{xxx}} }
\title{Chemia}

\begin{document}
\maketitle
%\tableofcontents

\section{Rozcieńczanie}

\begin{zad}
Jaką objętość wody należy dodać do 1 dm$^3$ 50\% roztworu, aby uzyskać 30\% roztwór?
\end{zad}


\begin{zad}
Jaką masę wody należy dodać do 0.5 kg roztworu o stężeniu 30\%, aby uzyskać roztwór o stężeniu 20\%?
\end{zad}

\begin{zad}
Ile razy należy rozcieńczyć roztwór 40\% aby uzyskać roztwór o dwa razy niższym stężeniu?
\end{zad}

\section{Obliczenia stechiometryczne}

\begin{zad}
\textbf{Mieszaniną piorunującą} nazywamy mieszaninę tlenu i wodoru w proporcjach molowych 1:2. Po zainicjowaniu reakcji zachodzi synteza w wyniku której powstaje woda i wydzielane są duże ilości energii.

\begin{enumerate}
\item Rozpisz równanie reakcji syntezy wodoru i tlenu.
\item W jakich proporcjach masowych należy zmieszać wodór i tlen aby reakcja zaszła całkowicie?
\item Ile potrzebujemy tlenu, aby całkowicie spalić 20 g wodoru?
\end{enumerate}
\end{zad}

\begin{zad}
Mieszaninę tlenku żelaza (III) z glinem w proporcjach stechiometrycznych określa się w pirotechnice jako \textbf{termit}. Reakcja redukcji-utleniania - nazywana reakcją \textbf{redoks} - zachodzi z wydzielaniem dużych ilości energii - osiągana jest temperatura sięgająca $2500^\circ$ C.

\begin{enumerate}
\item Rozpisz równanie reakcji redukcji-utleniania.
\item W jakich proporcjach masowych należy zmieszać substraty aby reakcja zaszła z maksymalną wydajnością?
\item Ile gramów glinu potrzebujemy, jeśli do reakcji wzięto 100 gramów tlenku żelaza III?
\end{enumerate}
\end{zad}

\begin{zad}
Amoniak $(NH_3)$ uzyskiwany jest poprzez metoę Habera-Boscha bezpośrednio z wodoru i azotu. Gazy spręża się do ciśnienia kilkuset atmosfer, po czym w obecności katalizatora powstaje amoniak. Reakcja jest równowagowa - substraty i produkty współistnieją w mieszaninie poreakcyjnej w ustalonych proporcjach.

\begin{enumerate}
\item Rozpisz równanie reakcji syntezy amoniaku.
\item Oblicz w jakich proporcjach masowych muszą zostać wzięte azot i wodór.
\item Ile gramów wodoru i azotu zostanie zużyte do wyprodukowania 10 kg amoniaku?
\end{enumerate}
\end{zad}

\begin{zad}
(*) Jedną z metod otrzymywania siarkowodoru jest bezpośrednia synteza siarki z żelazem, po czym do otrzymanego siarczku żelaza (III) dodaje się kwasu solnego.
\begin{enumerate}
\item Rozpisz równania obydwu reakcji.
\item W jakich proporcjach masowych należy zmieszać żelazo z siarką, aby reakcja zaszła całkowicie?
\item Ile gramów siarki należy dodać do 30 gramów żelaza aby przeprowadzić reakcję z najlepszą wydajnością?
\item Uzyskany siarczek żelaza rozpuszczono w kwasie solnym. Oblicz ile gramów czystego kwasu solnego zostanie zużyte na przeprowadzenie reakcji.
\item Do przeprowadzenia reakcji wzięto 10\% roztwór kwasu solnego (gęstość $1 g/cm^3$).  Jaką objętość roztworu należy użyć?
\end{enumerate}
\end{zad}


\end{document}
