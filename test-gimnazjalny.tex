\documentclass[11pt]{article}
\usepackage[utf8]{inputenc}
\usepackage[margin=0.1in]{geometry}
\usepackage{polski} 
\geometry{a4paper}
\usepackage{graphicx}
\usepackage{booktabs}
\usepackage{array}
\usepackage{paralist}
\usepackage{verbatim}
\usepackage{subfig}
\usepackage{amsmath}
\usepackage{float}
\usepackage{amsthm}
\usepackage{amssymb}
\usepackage{amsfonts}
\usepackage{float}
\usepackage{thmtools}
\usepackage{pgffor}
\usepackage[final]{pdfpages}
\theoremstyle{definition}
\newtheorem{zad}{Zadanie}
\numberwithin{zad}{section}
\newtheorem{theorem}{Twierdzenie}
\newtheorem{definition}{Definicja}
\newtheorem{lemma}{Lemat}
\renewcommand*{\proofname}{Rozwiązanie}
\usepackage[section]{placeins}
\usepackage{fancyhdr}
\pagestyle{fancy}
\renewcommand{\headrulewidth}{0pt}
\usepackage{sectsty}
\allsectionsfont{\sffamily\mdseries\upshape}
\usepackage[nottoc,notlof,notlot]{tocbibind}
\usepackage[titles,subfigure]{tocloft}
\renewcommand{\cftsecfont}{\rmfamily\mdseries\upshape}
\renewcommand{\cftsecpagefont}{\rmfamily\mdseries\upshape}
\newcommand{\blank}{\framebox{\phantom{xxx}} }
\title{Test gimnazjalny}

\begin{document}
\newcommand{\obraz}[1]{
\begin{figure}[H]
\centering
\includegraphics[width=0.7\linewidth]{./test-gim/#1.png}
\end{figure}
}

\obraz{01}
\obraz{02}
\obraz{03}
\obraz{04}
\obraz{05}
\obraz{06}
\obraz{07}
\obraz{08}
\obraz{09}
\obraz{10}
\obraz{11}
\obraz{12}
\obraz{13}
\obraz{14}
\obraz{15}
\obraz{16}
\obraz{17}
\obraz{18}
\obraz{19}
\obraz{20}
\obraz{21}
\obraz{22}
\obraz{23}
\obraz{24}
\obraz{25}
\obraz{26}
\obraz{27}
\obraz{28}
\obraz{29}
\obraz{30}
\obraz{31}
\obraz{32}
\obraz{33}
\obraz{34}
\obraz{35}
\obraz{36}
\obraz{37}
\obraz{38}
\obraz{39}
\obraz{40}
\obraz{41}
\obraz{42}
\obraz{43}
\obraz{44}
\obraz{45}
\obraz{46}
\obraz{47}
\obraz{48}
\obraz{49}
\obraz{50}
\obraz{51}
\obraz{52}
\obraz{53}

\end{document}
